% Пользовательские команды

% Команда определения заголовков разделов, которые без нумерации и по середине страницы
\newcommand{\csection}[1]{
	{
			\titleformat{\section}[block]{\centering}{\thesection}{1em}{}
			\phantomsection
			\section*{#1}
			\addcontentsline{toc}{section}{#1}
		}
}


\newcounter{russianletters}
\renewcommand{\therussianletters}{\ifcase\value{russianletters}\or А\or Б\or В\or Г\or Д\or Е\or Ж\or З\or И\or Й\or К\or Л\or М\or Н\or О\or П\or Р\or С\or Т\or У\or Ф\or Х\or Ц\or Ч\or Ш\or Щ\or Ъ\or Ы\or Ь\or Э\or Ю\or Я\fi}
% Приложение
% 1 необязательный аргумент - характер приложения (По умолчанию - обязательное)
% 2 аргумент - название приложения
\newcommand{\docappendix}[2][обязательное]{
	\refstepcounter{russianletters}
	{
		\newpage
		\begin{center}
			\phantomsection
			\addcontentsline{toc}{section}{Приложение \therussianletters. #2}
			Приложение \therussianletters

			(#1)

			#2
		\end{center}
		\vspace{1em}
	}
}

% Ссылка на элемент библиографического списка
\newcommand{\refref}[1]{\hyperref[#1]{[\ref*{#1}]}}
