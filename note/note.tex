%! TEX TS-program = xelatex

\documentclass[a4paper,10pt]{article}

% Подключение библиотек
\usepackage{geometry}
\usepackage{lipsum}
\usepackage{fancyhdr}
\usepackage{fontspec}
\usepackage{setspace}
\usepackage{ulem}
\usepackage{indentfirst}
\usepackage[english, russian]{babel}
\usepackage[hidelinks]{hyperref}
\usepackage{graphicx}
\usepackage{amsmath}
\usepackage{totcount}
\usepackage{calc}
\usepackage{tabularx}
\usepackage{ifthen}

% Установка пути картинок
\graphicspath{ {./images/} {./images.example/} }

% Установка базового шрифта (Требуется XeLatex)
\setmainfont{Times New Roman}

% Подключение файлов. В них возможны подключения библиотек. Поэтому выше список используемых  библиотек не полон.
% Файл с константами документа
\IfFileExists{./constants.tex}{
    \input{./constants.tex}
}{
    % Константы

% Тема
\newcommand{\topic}{Разработка машины времени}
% Фамилия автора документа
\newcommand{\authorsecondname}{Волков}
% Фамилия с инициалами
\newcommand{\authorwithinitials}{
	\authorsecondname~М.~В.}


% ТПЖА
\newcommand{\tpga}{TПЖА 09.03.01.514~ПЗ}
% Направление
\newcommand{\specialization}{09.03.01 - Информатика и вычислительная техника}
% Профиль
\newcommand{\profile}{Программное и аппаратное обеспечение вычислительной техники}

% Студент. Группа
\newcommand{\studentgroup}{студент гр.ИВТб-4301-04-00}

% Заведующая кафедрой
\newcommand{\headofdepartment}{Долженкова М. Л.}

% Руководитель
\newcommand{\supervisor}{Долженкова М. Л.}

% Звание руководителя
\newcommand{\supervisorrank}{\\ к.т.н., доцент, зав. кафедрой ЭВМ}

% Указание консультанта
\newboolean{consultantexists}
\setboolean{consultantexists}{True}

% Консультант
\newcommand{\consultant}{Кошкин О. В.}

% Звание консультанта
\newcommand{\consultantrank}{преподаватель кафедры ЭВМ}


% Нормоконтролер
\newcommand{\norminspector}{Скворцов А. А.}

% Звание нормоконтролера
\newcommand{\norminspectorrank}{к.т.н., доцент}

% Количество плакатов
\newcommand{\numberofposters}{8}

% Рамки %

% Кафедра и группа
\newcommand{\departmentandgroupinframe}{Кафедра ЭВМ Группа ИВТ-41}

% Разработал
\newcommand{\authorinframe}{Волков}

% Проверяющий
\newcommand{\inspectorinframe}{Долженкова}

% Нормоконтролер
\newcommand{\norminspectorinframe}{Скворцов}

% Утверждающий
\newcommand{\approverinframe}{Долженкова}

}
% Файл с рамками
% Определение рамок

\usepackage{tikz}

\usepackage{xkeyval}





\makeatletter

\define@key{subsequenttable}{tpga}[]{\def\st@tpga{#1}}
\define@key{subsequenttable}{page}[]{\def\st@page{#1}}


\newcommand{\subsequenttable}[1][]{
	\setkeys{subsequenttable}{
		tpga,
		page
	}
	\setkeys{subsequenttable}{#1}
	\itshape
	\small
	\arial
	\begin{tikzpicture}[remember picture, overlay]

		\draw[black, ultra thick]
		([shift={(-5mm, 20mm)}] current page.south east)
		--
		+(-185mm, 0)

		([shift={(-190mm, 10mm)}] current page.south east)
		--
		+(65mm, 0)

		([shift={(-5mm, 5mm)}] current page.south east)
		+(0, 8mm) -- +(-10mm, 8mm)
		+(-10mm, 0) -- +(-10mm, 15mm)

		([shift={(-190mm, 5mm)}] current page.south east)
		-- +(0, 15mm)
		++(7mm, 0) -- +(0, 15mm)
		++(10mm, 0) -- +(0, 15mm)
		++(23mm, 0) -- +(0, 15mm)
		++(15mm, 0) -- +(0, 15mm)
		++(10mm, 0) -- +(0, 15mm)
		;

		\draw[black, thick]
		([shift={(-190mm, 15mm)}] current page.south east)
		-- +(65mm, 0)
		;

		\draw[anchor=mid]
		([shift={(-190mm, 5mm)}] current page.south east)
		++(0, 2.3mm)
		+(3.5mm, 0)
		node {Изм.}

		++(7mm, 0)
		+(5mm, 0)
		node {Лист}

		++(10mm, 0)
		+(11.5mm, 0)
		node {№ докум.}

		++(23mm, 0)
		+(7.5mm, 0)
		node {Подп.}

		++(15mm, 0)
		+(5mm, 0)
		node {Дата}

		([shift={(-5mm, 13mm)}] current page.south east)
		+(-5mm, 3.3mm)
		node {Лист}
		;

		\normalsize
		\draw
		([shift={(-10mm, 9mm)}] current page.south east)
		node {\st@page}
		;

		\huge
		\draw
		([shift={(-70mm, 5mm)}] current page.south east)
		+(0mm, +7.5mm)
		node {\st@tpga}
		;

	\end{tikzpicture}
}


\makeatother


\makeatletter

% Определение команды с ключами и значениями
\define@key{titletable}{creator}[]{\def\tt@creator{#1}}
\define@key{titletable}{inspector}[]{\def\tt@inspector{#1}}
\define@key{titletable}{norminspector}[]{\def\tt@norminspector{#1}}
\define@key{titletable}{approver}[]{\def\tt@approver{#1}}
\define@key{titletable}{page}[]{\def\tt@page{#1}}
\define@key{titletable}{total}[]{\def\tt@total{#1}}
\define@key{titletable}{departmentandgroup}[]{\def\tt@departmentandgroup{#1}}
\define@key{titletable}{tpga}[]{\def\tt@tpga{#1}}
\define@key{titletable}{name}[]{\def\tt@name{#1}}



% Таблица рамки заглавной страницы
\newcommand{\titletable}[1][]{
	\setkeys{titletable}{
		creator,
		inspector,
		norminspector,
		approver,
		page,
		total,
		departmentandgroup,
		tpga,
		name,
	}
	\setkeys{titletable}{#1}
	\itshape
	\small
	\begin{tikzpicture}[remember picture, overlay]

		\draw[black, ultra thick]
		([shift={(-5mm, 60mm)}] current page.south east)
		--
		+(-185mm, 0)
		++(0, -15mm)
		--
		+(-120mm, 0)
		++(0, -5mm)
		--
		+(-50mm, 0)
		++(0, -15mm)
		--
		+(-50mm, 0)
		++(0, -5mm)
		--
		++(-120mm, 0)
		++(0, 15mm)
		--
		+(-65mm, 0)
		++(0, 5mm)
		--
		++(-65mm, 0)
		++(0, 20mm)
		--
		+(0, -55mm)
		++(7mm, 0)
		--
		+(0, -25mm)
		++(10mm, 0)
		--
		+(0, -55mm)
		++(23mm, 0)
		--
		+(0, -55mm)
		++(15mm, 0)
		--
		+(0, -55mm)
		++(10mm, 0)
		--
		+(0, -55mm)

		++(70mm, -15mm)
		--
		+(0, -40mm)
		++(15mm, 0)
		--
		+(0, -20mm)
		+(5mm, -20mm)
		--
		+(5mm, -25mm)
		++(17mm, 0)
		--
		+(0, -20mm)
		;


		\draw[black, thick]
		([shift={(-190mm, 55mm)}] current page.south east)
		--
		+(65mm, 0)
		++(0, -5mm)
		--
		+(65mm, 0)
		++(0, -5mm)
		--
		+(65mm, 0)
		++(0, -15mm)
		--
		+(65mm, 0)
		++(0, -5mm)
		--
		+(65mm, 0)
		++(0, -5mm)
		--
		+(65mm, 0)
		++(0, -5mm)
		--
		+(65mm, 0)
		++(0, -5mm)
		--
		++(65mm, 0)

		++(75mm, 15mm)
		--
		+(0, 15mm)
		++(5mm, 0)
		--
		+(0, 15mm)
		++(5mm, 0)
		--
		+(0, 15mm)


		;

		\draw[anchor=mid]
		([shift={(-190mm, 40mm)}] current page.south east)
		++(0, -2.7mm)
		+(3.5mm, 0)
		node {Изм.}

		+(12mm, 0)
		node {Лист}

		+(28.5mm, 0)
		node {№ докум.}

		+(47.5mm, 0)
		node {Подп.}

		+(60mm, 0)
		node {Дата}

		+(8.5mm, -5mm)
		node [
				text width = 15mm,
				align = left
			] {Разраб.}

		+(8.5mm, -10mm)
		node [
				text width = 15mm,
				align = left
			]{Пров.}

		+(8.5mm, -15mm)
		node [
				text width = 15mm,
				align = left
			] {Т. контр.}

		+(8.5mm, -25mm)
		node [
				text width = 15mm,
				align = left
			] {Н. контр.}

		+(8.5mm, -30mm)
		node [
				text width = 15mm,
				align = left
			] {Утв.}

		++(28.5mm, -5mm)
		+(0, 0mm)
		node [
				text width = 21mm,
				align = left
			] {\tt@creator}

		+(0, -5mm)
		node [
				text width = 21mm,
				align = left
			] {\tt@inspector}

		+(0, -20mm)
		node [
				text width = 21mm,
				align = left
			] {\tt@norminspector}

		+(0, -25mm)
		node [
				text width = 21mm,
				align = left
			] {\tt@approver}



		([shift={(-5mm, 25mm)}] current page.south east)
		++(0, -2.7mm)
		+(-15mm, 0)
		node {Листов~\tt@total}

		+(-40mm, 0)
		node {Лист~\tt@page}

		++(0, 20mm)
		+(-9mm, 0)
		node {Масштаб}
		+(-26.5mm, 0)
		node {Масса}
		+(-42.5mm, 0)
		node {Лит.}
		;


		%\upshape
		\huge
		\draw
		([shift={(-5mm, 45mm)}] current page.south east)
		+(-60mm, 7.5mm)
		node {\tt@tpga}
		;

		\large
		\draw
		([shift={(-5mm, 5mm)}] current page.south east)
		+(-25mm, 7.5mm)
		node [
				anchor = center,
				text width = 50mm,
				align = center
			]{\tt@departmentandgroup}
		;

		\LARGE
		\draw
		([shift={(-55mm, 20mm)}] current page.south east)
		+(-35mm, 12.5mm)
		node [
				anchor = center,
				text width = 70mm,
				align = center
			]{\tt@name}
		;

	\end{tikzpicture}
}
\makeatother






\newcommand{\simpleframe}{
	\itshape
	\small
	\begin{tikzpicture}[remember picture, overlay]

		\draw[black, ultra thick]

		([shift={(20mm, 5mm)}] current page.south west)
		--
		([shift={(20mm, -5mm)}] current page.north west)
		--
		([shift={(-5mm, -5mm)}] current page.north east)
		--
		([shift={(-5mm, 5mm)}] current page.south east)
		-- cycle;

	\end{tikzpicture}
}


% Верхняя надпись для альбомной страницы
\newcommand{\topinsсriptionforlandscape}[1][]{
	\itshape
	\begin{tikzpicture}[remember picture, overlay]

		\draw[black, ultra thick]

		([shift={(90mm, -5mm)}] current page.north west)
		--
		++(0, -14mm)
		--
		+(-70mm, 0);


		\Large
		\draw
		([shift={(55mm, -12mm)}] current page.north west)
		node[rotate=180]{#1};


	\end{tikzpicture}
}



% Верхняя надпись для книжной страницы
\newcommand{\topinsсriptionforportrait}[1][]{
	\itshape
	\begin{tikzpicture}[remember picture, overlay]

		\draw[black, ultra thick]

		([shift={(-5mm, -75mm)}] current page.north east)
		--
		++(-14mm, 0mm)
		--
		+(0, 70mm);


		\Large
		\draw
		([shift={(-12mm, -40mm)}] current page.north east)
		node[rotate=90]{#1};


	\end{tikzpicture}
}


\newcommand{\paperformat}[2][Формат]{
	\begin{tikzpicture}[remember picture, overlay]

		\draw[black, ultra thick]
		([shift={(-30mm, 2.3mm)}] current page.south east)
		node{#1 #2};

	\end{tikzpicture}
}


\makeatletter

% Определение команды с ключами и значениями
\define@key{booktitleframe}{creator}[]{\def\bktf@creator{#1}}
\define@key{booktitleframe}{inspector}[]{\def\bktf@inspector{#1}}
\define@key{booktitleframe}{norminspector}[]{\def\bktf@norminspector{#1}}
\define@key{booktitleframe}{approver}[]{\def\bktf@approver{#1}}
\define@key{booktitleframe}{page}[]{\def\bktf@page{#1}}
\define@key{booktitleframe}{total}[]{\def\bktf@total{#1}}
\define@key{booktitleframe}{departmentandgroup}[]{\def\bktf@departmentandgroup{#1}}
\define@key{booktitleframe}{tpga}[]{\def\bktf@tpga{#1}}
\define@key{booktitleframe}{topic}[]{\def\bktf@topic{#1}}
\define@key{booktitleframe}{paperformat}[]{\def\bktf@paperformat{#1}}


% Рамка заглавной страницы бумажного формата
\newcommand{\booktitleframe}[1][]{
\setkeys{booktitleframe}{
	creator,
	inspector,
	norminspector,
	approver,
	page,
	total,
	departmentandgroup,
	tpga,
	topic,
	paperformat,
}
\setkeys{booktitleframe}{#1}
\simpleframe
\titletable[
	tpga=\bktf@tpga,
	departmentandgroup=\bktf@departmentandgroup,
	topic=\bktf@topic,
	creator=\bktf@creator,
	inspector=\bktf@inspector,
	norminspector=\bktf@norminspector,
	approver=\bktf@approver,
	page=\bktf@page,
	total=\bktf@total,
]
\topinsсriptionforbook[\bktf@tpga]
\paperformat{\bktf@paperformat}
}


\makeatother




\makeatletter

% Определение команды с ключами и значениями
\define@key{landscapetitleframe}{creator}[]{\def\lstf@creator{#1}}
\define@key{landscapetitleframe}{inspector}[]{\def\lstf@inspector{#1}}
\define@key{landscapetitleframe}{norminspector}[]{\def\lstf@norminspector{#1}}
\define@key{landscapetitleframe}{approver}[]{\def\lstf@approver{#1}}
\define@key{landscapetitleframe}{page}[]{\def\lstf@page{#1}}
\define@key{landscapetitleframe}{total}[]{\def\lstf@total{#1}}
\define@key{landscapetitleframe}{departmentandgroup}[]{\def\lstf@departmentandgroup{#1}}
\define@key{landscapetitleframe}{tpga}[]{\def\lstf@tpga{#1}}
\define@key{landscapetitleframe}{topic}[]{\def\lstf@topic{#1}}
\define@key{landscapetitleframe}{paperformat}[]{\def\lstf@paperformat{#1}}


% Рамка последующих страниц бумажного формата
\newcommand{\landscapetitleframe}[1][]{
\setkeys{landscapetitleframe}{
	creator,
	inspector,
	norminspector,
	approver,
	page,
	total,
	departmentandgroup,
	tpga,
	topic,
	paperformat,
}
\setkeys{landscapetitleframe}{#1}
\simpleframe
\titletable[
	tpga=\lstf@tpga,
	departmentandgroup=\lstf@departmentandgroup,
	topic=\lstf@topic,
	creator=\lstf@creator,
	inspector=\lstf@inspector,
	norminspector=\lstf@norminspector,
	approver=\lstf@approver,
	page=\lstf@page,
	total=\lstf@total,
]
\topinsсriptionforlandscape[\lstf@tpga]
\paperformat{\lstf@paperformat}
}


\makeatother


% Файл, в котором содержаться пользовательские команды
\newcommand{\HUGE}{\fontsize{40}{50}\selectfont}

% Общие натройки стилей
\input{./styles/general.tex}
% Файл настроек / стилизации таблицы содержания ("Содержание")
% Изменение стилей таблицы содержимого
\usepackage{tocloft}

\setcounter{tocdepth}{2}

\renewcommand{\cftsecleader}{\cftdotfill{\cftdotsep}}
\renewcommand{\cftdotsep}{1}
\cftsetrmarg{0pt}
\renewcommand{\cfttoctitlefont}{}


\renewcommand{\cftsecpagefont}{}
\renewcommand{\cftsubsecpagefont}{}
\renewcommand{\cftsubsubsecpagefont}{}
\renewcommand{\cftparafont}{}

%\newcommand{\secnumwidth}{1cm}
%\newcommand{\subsecnumwidth}{1.5cm}
%\newcommand{\subsubsecnumwidth}{2cm}
%\newcommand{\paranumwidth}{2cm}


\setlength{\cftsecnumwidth}{2em}
\setlength{\cftsubsecnumwidth}{3em}
\setlength{\cftsubsubsecnumwidth}{4em}
\setlength{\cftparanumwidth}{5em}

\newcommand{\secindent}{0em}
\newcommand{\subsecindent}{1em}
\newcommand{\subsubsecindent}{2em}
\newcommand{\paraindent}{3em}

\setlength{\cftsecindent}{-\cftsecnumwidth}
\setlength{\cftsubsecindent}{-\cftsubsecnumwidth}
\setlength{\cftsubsubsecindent}{-\cftsubsubsecnumwidth}
\setlength{\cftparaindent}{-\cftparanumwidth}

\renewcommand{\cftsecfont}{\hspace{\cftsecnumwidth}\hspace{\secindent}}
\renewcommand{\cftsubsecfont}{\hspace{\cftsubsecnumwidth}\hspace{\subsecindent}}
\renewcommand{\cftsubsubsecfont}{\hspace{\cftsubsubsecnumwidth}\hspace{\subsubsecindent}}
\renewcommand{\cftparafont}{\hspace{\cftparanumwidth}\hspace{\paraindent}}


\setlength{\cftbeforesecskip}{0.5em}
\setlength{\cftbeforesubsecskip}{0.5em}
\setlength{\cftbeforesubsubsecskip}{0.5em}
\setlength{\cftbeforeparaskip}{0.5em}




% Файл стилизации названий разделов
% Изменение стилей названий разделов
\usepackage{titlesec}

\setcounter{secnumdepth}{4}

\titleformat{\section}[block]{\hspace{\parindent}}{\thesection}{1em}{}
\titleformat{\subsection}[block]{\hspace{\parindent}}{\thesubsection}{1em}{}
\titleformat{\subsubsection}[block]{\hspace{\parindent}}{\thesubsubsection}{1em}{}
\titleformat{\paragraph}[block]{\hspace{\parindent}}{\theparagraph}{1em}{}


\titlespacing{\section}{0pt}{2em}{2em}
\titlespacing{\subsection}{0pt}{2em}{2em}
\titlespacing{\subsubsection}{0pt}{0em}{0em}
\titlespacing{\paragraph}{0pt}{0em}{0em}

% Файл стилизации описаний рисунков и таблиц
% Изменение стилей описаний рисунков, таблиц...
\usepackage{caption}

\DeclareCaptionLabelSeparator{custom}{ -- }
\DeclareCaptionLabelFormat{flformat}{Рисунок #2}
\DeclareCaptionLabelFormat{tlformat}{Таблица #2}


\captionsetup[figure]{
	font=Large,
	labelsep=custom,
	labelformat=flformat,
	justification=centering,
	margin=1cm,
	aboveskip=0.5cm,
	belowskip=0.5cm,
}


\captionsetup[table]{
	format=plain,
	font=Large,
	labelsep=custom,
	labelformat=tlformat,
	singlelinecheck=false,
	margin=0pt,
	%margin={\docparindent, 0pt},
	aboveskip=1em,
	belowskip=0.5cm,
}

% Файл стилизации списков
% Изменение стилей списков

\usepackage{enumitem}



\newcommand{\labelv}{--}
% Отступ от label у itemize
\newcommand{\ilabelsep}{0.5em}
% Отступ от label y enumerate, level 1
\newcommand{\eilabelsep}{0.5em}
% Отступ от label у enumarate, level 2
\newcommand{\eiilabelsep}{0em}


\setlist[itemize]{
	itemsep=0pt,
	parsep=0pt,
	topsep=0pt,
	listparindent=\parindent,
	labelsep=\ilabelsep,
	label=\labelv,
	itemindent=\parindent+\ilabelsep+\labelwidth,
	leftmargin=0pt,
	align=left,
}


\setlist[enumerate,1]{
	label=\arabic*),
	ref=\arabic*,
	align=left,
	itemsep=0pt,
	parsep=0pt,
	topsep=0pt,
	labelwidth=\widthof{99)},
	labelsep=\eilabelsep,
	leftmargin=\parindent+\labelwidth+\labelsep,
}

\setlist[enumerate,2]{
	label=\theenumi.\arabic*),
	align=left,
	itemsep=0pt,
	parsep=0pt,
	topsep=0pt,
	labelsep=\eiilabelsep,
	labelwidth=\widthof{99.99)},
	leftmargin=\labelwidth+\labelsep,
}

\newlist{references}{enumerate}{1}
\setlist[references]{
	itemsep=0pt,
	parsep=0pt,
	topsep=0pt,
	labelsep=\ilabelsep,
	label=\arabic*.,
	ref=\arabic*,
	itemindent=\ilabelsep+\labelwidth,
	leftmargin=0pt,
	align=left,
}

% Файл стилизации кода
\input{./styles/listings.tex}

%\renewcommand{\cftchapleader}{\cftdotfill{\cftdotsep}}

% Настройка отступов от краёв страницы до текста
\geometry{top = 15mm, left = 25mm, right = 10mm, bottom = 15mm}

% Убираем линию у верхнего колонтитула
\renewcommand{\headrulewidth}{0pt}

% Определяем стиль главной рамки
\fancypagestyle{mainframe} {
    \fancyhf{}
    \fancyhead[L]{
        \mainframe
        {\tpga}
        {\departmentandgroupinframe}{
        \topic
        }
        {\authorinframe}
        {\inspectorinframe}
        {\norminspectorinframe}
        {\approverinframe}
        {\thepage}
        {\total{page}}
    }

}
% Определяем стиль рамки для страниц
\fancypagestyle{pageframe} {
    \fancyhf{}
    \fancyhead[L]{
        \pageframe
        {\tpga}
        {\thepage}
    }
}


%\setlength{\textfloatsep}{1cm plus 0.5cm minus 0.5cm}

% Межстрочный интервал, равный 1.5
\onehalfspacing

% Устанавливаем отступ абзаца (Значение берется из файла констант)
\setlength{\parindent}{\docparindent}

% Выравнивание по ширине 
% (Если не влезает текст, то увеличивает отступы между словами)
\sloppy

% Регистрация счётчиков для подсчёта разных объектов документа(страниц, формул, таблиц...)
\regtotcounter{page}
\regtotcounter{equation}
\regtotcounter{figure}
\regtotcounter{table}
\regtotcounter{appendix}

% Подключение файла пользовательских стилей, команд и т.д.
\IfFileExists{custom.tex}{
    \input{custom.tex}
}{
    % Здесь указываются пользовательские стили, команды...
% Также возможно преопределение стилей, команд


% Команда изменения стиля темы в рамке.
% В данном примере размер шрифта - large.
% Если данная команда не будет указана, то по умолчанию 
% для темы в рамке будет указан размер шрифта - \large.
% Если вам не нужно менять стиль темы, то закомментируйте или удалите данную команду.
\newcommand{\topicstyleinframe}{\large}

}


% Начало документа
\begin{document}
% Меняем название заголовка таблицы содержимого
\renewcommand{\contentsname}{\hfill Содержание \hfill}
% Устанавливаем размер шрифта 14pt

\Large
% Устанавливаем стиль пустой, чтобы убрать дефолтную нумерацию страниц
\pagestyle{empty}
% Подключаем титульник
% Титульник не изменяет счётчик страниц, он как бы не участвует в нумерации страниц,
% поэтому нумерация страниц начнётся со страницы "Реферат". 
% В итоге как бы реферат не будет участвовать в нумерации, что нам и нужно
\IfFileExists{./pages/title.tex} {
	
\makeatletter

% Определение команды с ключами и значениями
\define@key{titletable}{creator}[]{\def\tt@creator{#1}}
\define@key{titletable}{inspector}[]{\def\tt@inspector{#1}}
\define@key{titletable}{norminspector}[]{\def\tt@norminspector{#1}}
\define@key{titletable}{approver}[]{\def\tt@approver{#1}}
\define@key{titletable}{page}[]{\def\tt@page{#1}}
\define@key{titletable}{total}[]{\def\tt@total{#1}}
\define@key{titletable}{departmentandgroup}[]{\def\tt@departmentandgroup{#1}}
\define@key{titletable}{tpga}[]{\def\tt@tpga{#1}}
\define@key{titletable}{name}[]{\def\tt@name{#1}}



% Таблица рамки заглавной страницы
\newcommand{\titletable}[1][]{
	\setkeys{titletable}{
		creator,
		inspector,
		norminspector,
		approver,
		page,
		total,
		departmentandgroup,
		tpga,
		name,
	}
	\setkeys{titletable}{#1}
	\itshape
	\small
	\begin{tikzpicture}[remember picture, overlay]

		\draw[black, ultra thick]
		([shift={(-5mm, 60mm)}] current page.south east)
		--
		+(-185mm, 0)
		++(0, -15mm)
		--
		+(-120mm, 0)
		++(0, -5mm)
		--
		+(-50mm, 0)
		++(0, -15mm)
		--
		+(-50mm, 0)
		++(0, -5mm)
		--
		++(-120mm, 0)
		++(0, 15mm)
		--
		+(-65mm, 0)
		++(0, 5mm)
		--
		++(-65mm, 0)
		++(0, 20mm)
		--
		+(0, -55mm)
		++(7mm, 0)
		--
		+(0, -25mm)
		++(10mm, 0)
		--
		+(0, -55mm)
		++(23mm, 0)
		--
		+(0, -55mm)
		++(15mm, 0)
		--
		+(0, -55mm)
		++(10mm, 0)
		--
		+(0, -55mm)

		++(70mm, -15mm)
		--
		+(0, -40mm)
		++(15mm, 0)
		--
		+(0, -20mm)
		+(5mm, -20mm)
		--
		+(5mm, -25mm)
		++(17mm, 0)
		--
		+(0, -20mm)
		;


		\draw[black, thick]
		([shift={(-190mm, 55mm)}] current page.south east)
		--
		+(65mm, 0)
		++(0, -5mm)
		--
		+(65mm, 0)
		++(0, -5mm)
		--
		+(65mm, 0)
		++(0, -15mm)
		--
		+(65mm, 0)
		++(0, -5mm)
		--
		+(65mm, 0)
		++(0, -5mm)
		--
		+(65mm, 0)
		++(0, -5mm)
		--
		+(65mm, 0)
		++(0, -5mm)
		--
		++(65mm, 0)

		++(75mm, 15mm)
		--
		+(0, 15mm)
		++(5mm, 0)
		--
		+(0, 15mm)
		++(5mm, 0)
		--
		+(0, 15mm)


		;

		\draw[anchor=mid]
		([shift={(-190mm, 40mm)}] current page.south east)
		++(0, -2.7mm)
		+(3.5mm, 0)
		node {Изм.}

		+(12mm, 0)
		node {Лист}

		+(28.5mm, 0)
		node {№ докум.}

		+(47.5mm, 0)
		node {Подп.}

		+(60mm, 0)
		node {Дата}

		+(8.5mm, -5mm)
		node [
				text width = 15mm,
				align = left
			] {Разраб.}

		+(8.5mm, -10mm)
		node [
				text width = 15mm,
				align = left
			]{Пров.}

		+(8.5mm, -15mm)
		node [
				text width = 15mm,
				align = left
			] {Т. контр.}

		+(8.5mm, -25mm)
		node [
				text width = 15mm,
				align = left
			] {Н. контр.}

		+(8.5mm, -30mm)
		node [
				text width = 15mm,
				align = left
			] {Утв.}

		++(28.5mm, -5mm)
		+(0, 0mm)
		node [
				text width = 21mm,
				align = left
			] {\tt@creator}

		+(0, -5mm)
		node [
				text width = 21mm,
				align = left
			] {\tt@inspector}

		+(0, -20mm)
		node [
				text width = 21mm,
				align = left
			] {\tt@norminspector}

		+(0, -25mm)
		node [
				text width = 21mm,
				align = left
			] {\tt@approver}



		([shift={(-5mm, 25mm)}] current page.south east)
		++(0, -2.7mm)
		+(-15mm, 0)
		node {Листов~\tt@total}

		+(-40mm, 0)
		node {Лист~\tt@page}

		++(0, 20mm)
		+(-9mm, 0)
		node {Масштаб}
		+(-26.5mm, 0)
		node {Масса}
		+(-42.5mm, 0)
		node {Лит.}
		;


		%\upshape
		\huge
		\draw
		([shift={(-5mm, 45mm)}] current page.south east)
		+(-60mm, 7.5mm)
		node {\tt@tpga}
		;

		\large
		\draw
		([shift={(-5mm, 5mm)}] current page.south east)
		+(-25mm, 7.5mm)
		node [
				anchor = center,
				text width = 50mm,
				align = center
			]{\tt@departmentandgroup}
		;

		\LARGE
		\draw
		([shift={(-55mm, 20mm)}] current page.south east)
		+(-35mm, 12.5mm)
		node [
				anchor = center,
				text width = 70mm,
				align = center
			]{\tt@name}
		;

	\end{tikzpicture}
}
\makeatother



}{
	\begin{titlepage}

	\newcommand{\verticalspace}{1.5cm}


	\ifthenelse{\boolean{consultantexists}}{
		\renewcommand{\verticalspace}{1.2cm}}{}

	\newpage
	\singlespacing
	\vspace{0.8cm}
	\large
	%\setlength{\ULdepth}{1.8pt}

	\begin{center}
		\large
		%\fontsize{11pt}{13pt}
		\bfseries
		МИНИСТЕРСТВО НАУКИ И ВЫСШЕГО ОБРАЗОВАНИЯ РФ \\
		ФЕДЕРАЛЬНОЕ ГОСУДАРСТВЕННОЕ БЮДЖЕТНОЕ \\
		ОБРАЗОВАТЕЛЬНОЕ УЧРЕЖДЕНИЕ ВЫСШЕГО ОБРАЗОВАНИЯ \\
		«ВЯТСКИЙ ГОСУДАРСТВЕННЫЙ УНИВЕРСИТЕТ» \\
		ИНСТИТУТ МАТЕМАТИКИ И ИНФОРМАЦИОННЫХ СИСТЕМ \\
		ФАКУЛЬТЕТ АВТОМАТИКИ И ВЫЧИСЛИТЕЛЬНОЙ ТЕХНИКИ \\
		КАФЕДРА ЭЛЕКТРОННЫХ ВЫЧИСЛИТЕЛЬНЫХ МАШИН
	\end{center}

	\vspace{0.8cm}
	\begin{center}
		\textbf{Направление}
		\uline{\specialization}

		\small
		\textit{(код и наименование направления)}

		\large
		Профиль – \uline{\profile}
	\end{center}

	\vspace{0.8cm}
	\begin{flushright}
		{
			\onehalfspacing
			Допускаю к защите \\
			Заведующий кафедрой ЭВМ \\
		}
		\vspace{1mm}
		\uline{\hspace{3cm}} / \uline{\headofdepartment} / \\
		\vspace{1mm}
		\small
		\itshape
		(подпись) \hspace{1.8cm} (Ф.И.О.) \hspace{1.4cm}

	\end{flushright}

	\vspace{1.5cm}
	\begin{center}
		\huge
		\bfseries
		\topic
	\end{center}
	\vspace{0pt}
	\begin{center}
		Пояснительная записка выпускной квалификационной работы \\
		\tpga
	\end{center}

	\newcommand{\ulinesize}{2.5cm}
	\newcommand{\namesize}{3.7cm}

	\large
	\vspace{1cm}
	\noindent
	Разработал: \studentgroup
	\hfill
	\uline{\hspace{\ulinesize}}
	/\uline{\makebox[\namesize][l]{ \authorwithinitials}} /
	\uline{\hspace{\ulinesize}}

	\vspace{\verticalspace}
	\noindent
	Руководитель: \supervisorrank
	\hfill
	\uline{\hspace{\ulinesize}}
	/\uline{\makebox[\namesize][l]{ \supervisor}} /
	\uline{\hspace{\ulinesize}}

	\ifthenelse{\boolean{consultantexists}}{
		\vspace{\verticalspace}
		\noindent
		Консультант: \consultantrank
		\hfill
		\uline{\hspace{\ulinesize}}
		/\uline{\makebox[\namesize][l]{ \consultant}} /
		\uline{\hspace{\ulinesize}}}{}

	\vspace{\verticalspace}
	\noindent
	Нормоконтролер: \norminspectorrank
	\hfill
	\uline{\hspace{\ulinesize}}
	/\uline{\makebox[\namesize][l]{ \norminspector}} /
	\uline{\hspace{\ulinesize}}

	{
		\small
		\itshape
		\hfill
		\makebox[\ulinesize][c]{(подпись)}~
		\makebox[\namesize][c]{(Ф.И.О)}~
		\makebox[\ulinesize][c]{(дата)}
	}

	\begin{center}
		\vfill
		Киров \the\year
		\vspace{1cm}
	\end{center}


\end{titlepage}

}

% Меняем размер нижнего отступа до текста, чтобы текст не заезжал на главную рамку. 
% (Отступ текста до рамки 10mm)
% Меняем как бы на странице "Реферат", чтобы изменения были задействованы на странице 
% "Содержание". 
\addtolength{\textheight}{-40mm}

% Подключаем страницу "Реферат"
\IfFileExists{./pages/abstract.tex} {
	\include{pages/abstract.tex}
}{
	
{

\vspace{0.8cm}
\begin{center}
	Реферат
\end{center}

\vspace{1em}

\authorwithinitials\
\topic
:\
\mbox{\tpga}\
ВКР / ВятГУ, каф. ЭВМ; рук.
\supervisor – Киров, \the\year. –
Гр.ч. \numberofposters л. ф.А1;
ПЗ
\total{page} с.,
\total{figure} рис.,
\total{table} табл.,
\total{equation} форм.,
\ref*{ref:total} источников,
\total{appendix} прил.

\vspace{1.5em}

\IfFileExists{pages/abstract/tags.tex}{
	\input{pages/abstract/tags.tex}
}{
	\input{pages/abstract.example/tags.tex}
}

\vspace{1.5em}

\IfFileExists{pages/abstract/content.tex}{
	\input{pages/abstract/content.tex}
} {
	\input{pages/abstract.example/content.tex}
}

}

}

% Устанавливаем главную рамку для одной страницы
\tocloftpagestyle{mainframe}

% Устанавливаем рамку страниц, которая будет отрисовываться после главной
\pagestyle{pageframe}

% Меняем размер нижнего отступа до текста, чтобы текст до рамки страниц был 10mm.
% Рамка страниц имеет меньшую высоту содержимого нижней таблицы.
\addtolength{\textheight}{+25mm}
% Устанавливаем "Содержание", включающее в себя описание разделов документа
\tableofcontents\newpage

% Подключаем содержимое документа
\IfFileExists{./pages/content.tex} {
	\input{pages/content.tex}
}{
	
&   &  &  &  &  & \\
&   &  & Документация общая &  &  & \\
&   &  & Вновь разработанная &  &  & \\
&   &  &  &  &  & \\
&   &  & Пояснительная записка & 15 &  & \\
&   &  &  &  &  & \\
&   &  & Графическая часть &  &  & \\
&   &  & Вновь разработанная &  &  & \\
&   &  & Cхемы алгоритмов & 2 &  & Плакат \\
&   &  &  &  &  & \\
&   &  &  &  &  & \\
&   &  &  &  &  & \\
&   &  &  &  &  & \\
&   &  &  &  &  & \\
&   &  &  &  &  & \\
&   &  &  &  &  & \\
&   &  &  &  &  & \\
&   &  &  &  &  & \\
&   &  &  &  &  & \\
&   &  &  &  &  & \\
&   &  &  &  &  & \\
&   &  &  &  &  & \\
&   &  &  &  &  & \\

}


\end{document}
