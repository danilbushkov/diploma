%! TEX TS-program = xelatex

\documentclass[a4paper,10pt]{article}

% Подключение библиотек
\usepackage{geometry}
\usepackage{lipsum}
\usepackage{fancyhdr}
\usepackage{fontspec}
\usepackage{setspace}
\usepackage{ulem}
\usepackage{indentfirst}
\usepackage[english, russian]{babel}
\usepackage[hidelinks]{hyperref}
\usepackage{graphicx}
\usepackage{amsmath}
\usepackage{totcount}
\usepackage{calc}
\usepackage{tabularx}
\usepackage{ifthen}

% Установка пути картинок
\graphicspath{ {./images/} {./images.example/} }

% Установка базового шрифта (Требуется XeLatex)
\setmainfont{Times New Roman}

% Подключение файлов. В них возможны подключения библиотек. Поэтому выше список используемых  библиотек не полон.
% Файл с константами документа
\IfFileExists{./constants.tex}{
    % Константы


\newcommand{\topic}{Разработка машины времени}
% Фамилия автора документа
\newcommand{\authorsn}{Волков}
% Фамилия с инициалами
\newcommand{\authori}{
	\authorsn~М. В.}
% ТПЖА
\newcommand{\tpga}{TПЖА.09.03.01.514~ПЗ}

\newcommand{\supervisor}{Долженкова М. Л.}







% Автор в главной рамке
\newcommand{\authorinframe}{\authorsn}

% Размер первого отступа абзаца.
\newcommand{\docparindent}{12.5mm}

% Размер верхнего отступа фигуры
\newcommand{\toppaddingoffigure}{1cm}

%

}{
    % Константы

% ТПЖА
\newcommand{\tpga}{TПЖА 09.03.01.514}

% Тип документа
\newcommand{\doctype}{ДПЛ}

% Кафедра и группа
\newcommand{\departmentandgroupinframe}{Кафедра ЭВМ Группа ИВТ-41}

% Разработал
\newcommand{\authorinframe}{Волков}

% Проверяющий
\newcommand{\inspectorinframe}{Долженкова}

% Нормоконтролер
\newcommand{\norminspectorinframe}{Скворцов}

% Утверждающий
\newcommand{\approverinframe}{Долженкова}

}
% Файл с рамками
% Определение рамок
\usepackage{fontspec}
\newcommand{\arial}{\fontspec{Arial}}
\usepackage{tikz}
\newcommand{\mainframe}[9]{
	\itshape
	\small
	\arial
	\begin{tikzpicture}[remember picture, overlay]

		\draw[black, ultra thick]

		([shift={(20mm, 5mm)}] current page.south west)
		--
		([shift={(20mm, -5mm)}] current page.north west)
		--
		([shift={(-5mm, -5mm)}] current page.north east)
		--
		([shift={(-5mm, 5mm)}] current page.south east)
		-- cycle;
		\draw[black, ultra thick]
		([shift={(20mm, 45mm)}] current page.south west)
		--
		([shift={(-5mm, 45mm)}] current page.south east)
		([shift={(20mm, 35mm)}] current page.south west)
		--
		++(65mm, 0)
		([shift={(20mm, 30mm)}] current page.south west)
		--
		([shift={(-5mm, 30mm)}] current page.south east)
		++(0, -5mm)
		--
		+(-50mm, 0)
		++(0, -5mm)
		--
		+(-50mm, 0)

		([shift={(20mm, 45mm)}] current page.south west)
		++(7mm, 0)
		--
		+(0mm, -15mm)
		++(10mm, 0)
		--
		+(0, -40mm)
		++(23mm, 0)
		--
		+(0, -40mm)
		++(15mm, 0)
		--
		+(0, -40mm)
		++(10mm, 0)
		--
		+(0, -40mm)

		([shift={(-5mm, 30mm)}] current page.south east)
		++(-20mm, 0)
		--
		+(0, -10mm)
		++(-15mm, 0)
		--
		+(0, -10mm)
		++(-15mm, 0)
		--
		+(0, -25mm);

		\draw[black, thick]
		([shift={(-5mm, 30mm)}] current page.south east)
		++(-40mm, -5mm)
		--
		+(0, -5mm)
		++(-5mm, 0)
		--
		+(0, -5mm);
		\draw[black, thick]
		([shift={(20mm, 5mm)}] current page.south west)
		++(0, 5mm)
		--
		+(65mm, 0)
		++(0, 5mm)
		--
		+(65mm, 0)
		++(0, 5mm)
		--
		+(65mm, 0)
		++(0, 5mm)
		--
		+(65mm, 0)
		++(0, 15mm)
		--
		+(65mm, 0);
		\draw[anchor=mid]
		([shift={(20mm, 35mm)}] current page.south west)
		++(0, -2.7mm)
		+(3.5mm, 0)
		node {Изм.}

		+(12mm, 0)
		node {Лист}

		+(28.5mm, 0)
		node {№ докум.}

		+(47.5mm, 0)
		node {Подп.}

		+(60mm, 0)
		node {Дата}

		+(8.5mm, -5mm)
		node [
				text width = 15mm,
				align = left
			] {Разраб.}

		+(8.5mm, -10mm)
		node [
				text width = 15mm,
				align = left
			]{Пров.}

		+(8.5mm, -15mm)
		node [
				text width = 15mm,
				align = left
			] {Реценз.}

		+(8.5mm, -20mm)
		node [
				text width = 15mm,
				align = left
			] {Н. контр.}

		+(8.5mm, -25mm)
		node [
				text width = 15mm,
				align = left
			] {Утв.}





		([shift={(-5mm, 30mm)}] current page.south east)
		++(0, -2.7mm)
		+(-10mm, 0)
		node {Листов}

		+(-27.5mm, 0)
		node {Лист}

		+(-42.5mm, 0)
		node {Литера}

		++(0, -5mm)
		+(-10mm, 0)
		node {#9}

		+(-27.5mm, 0)
		node {#8}
		;


		\draw[anchor=mid]
		([shift={(37mm, 30mm)}] current page.south west)

		+(11.5mm, -2.7mm)
		node [
				text width = 21mm,
				align = left
			] {#4}

		+(11.5mm, -7.7mm)
		node [
				text width = 21mm,
				align = left
			] {#5}

		+(11.5mm, -17.7mm)
		node [
				text width = 21mm,
				align = left
			] {#6}

		+(11.5mm, -22.7mm)
		node [
				text width = 21mm,
				align = left
			] {#7}
		;

		%\upshape
		\huge
		\draw
		([shift={(-5mm, 30mm)}] current page.south east)
		+(-60mm, 7.5mm)
		node {#1}
		;

		\large
		\draw
		([shift={(-5mm, 5mm)}] current page.south east)
		+(-25mm, 7.5mm)
		node [
				anchor = center,
				text width = 50mm,
				align = center
			]{#2}
		;

		\large
		\draw
		([shift={(-55mm, 5mm)}] current page.south east)
		+(-35mm, 12.5mm)
		node [
				anchor = center,
				text width = 70mm,
				align = center
			]{#3}
		;

	\end{tikzpicture}
}




\newcommand{\pageframe}[2]{
	\itshape
	\small
	\arial
	\begin{tikzpicture}[remember picture, overlay]

		\draw[black, ultra thick]

		([shift={(20mm, 5mm)}] current page.south west)
		--
		([shift={(20mm, -5mm)}] current page.north west)
		--
		([shift={(-5mm, -5mm)}] current page.north east)
		--
		([shift={(-5mm, 5mm)}] current page.south east)
		-- cycle;
		\draw[black, ultra thick]
		([shift={(20mm, 20mm)}] current page.south west)
		--
		([shift={(-5mm, 20mm)}] current page.south east)

		([shift={(20mm, 10mm)}] current page.south west)
		--
		+(65mm, 0)

		([shift={(-5mm, 5mm)}] current page.south east)
		+(0, 8mm) -- +(-10mm, 8mm)
		+(-10mm, 0) -- +(-10mm, 15mm)

		([shift={(20mm, 5mm)}] current page.south west)
		++(7mm, 0) -- +(0, 15mm)
		++(10mm, 0) -- +(0, 15mm)
		++(23mm, 0) -- +(0, 15mm)
		++(15mm, 0) -- +(0, 15mm)
		++(10mm, 0) -- +(0, 15mm)
		;

		\draw[black, thick]
		([shift={(20mm, 15mm)}] current page.south west)
		-- +(65mm, 0)
		;

		\draw[anchor=mid]
		([shift={(20mm, 5mm)}] current page.south west)
		++(0, 2.3mm)
		+(3.5mm, 0)
		node {Изм.}

		++(7mm, 0)
		+(5mm, 0)
		node {Лист}

		++(10mm, 0)
		+(11.5mm, 0)
		node {№ докум.}

		++(23mm, 0)
		+(7.5mm, 0)
		node {Подп.}

		++(15mm, 0)
		+(5mm, 0)
		node {Дата}

		([shift={(-5mm, 13mm)}] current page.south east)
		+(-5mm, 3.3mm)
		node {Лист}
		;

		\normalsize
		\draw
		([shift={(-10mm, 9mm)}] current page.south east)
		node {#2}
		;

		\huge
		\draw
		([shift={(85mm, 5mm)}] current page.south west)
		+(55mm, +7.5mm)
		node {#1}
		;

	\end{tikzpicture}
}

% Файл, в котором содержаться пользовательские команды
% Пользовательские команды

% Команда определения заголовков разделов, которые без нумерации и по середине страницы
\newcommand{\csection}[1]{
	{
			\titleformat{\section}[block]{\centering}{\thesection}{1em}{}
			\phantomsection
			\section*{#1}
			\addcontentsline{toc}{section}{#1}
		}
}


\newcounter{russianletters}
\renewcommand{\therussianletters}{\ifcase\value{russianletters}\or А\or Б\or В\or Г\or Д\or Е\or Ж\or З\or И\or Й\or К\or Л\or М\or Н\or О\or П\or Р\or С\or Т\or У\or Ф\or Х\or Ц\or Ч\or Ш\or Щ\or Ъ\or Ы\or Ь\or Э\or Ю\or Я\fi}
% Приложение
% 1 необязательный аргумент - характер приложения (По умолчанию - обязательное)
% 2 аргумент - название приложения
\newcommand{\docappendix}[2][обязательное]{
	{
			\newpage
			\stepcounter{russianletters}
			\begin{center}
				\phantomsection
				\addcontentsline{toc}{section}{Приложение \therussianletters. #2}
				Приложение \therussianletters

				(#1)

				#2
			\end{center}
			\vspace{1em}
		}
}

% Ссылка на элемент библиографического списка
\newcommand{\refref}[1]{\hyperref[#1]{[\ref*{#1}]}}

% Общие натройки стилей

% Размер первого отступа абзаца.
\newcommand{\docparindent}{12.5mm}

% Размер верхнего отступа фигуры
\newcommand{\toppaddingoffigure}{1cm}

% Размер нижнего отступа таблицы
\newcommand{\bottompaddingoftable}{0.5cm}


% Файл настроек / стилизации таблицы содержания ("Содержание")
% Изменение стилей таблицы содержимого
\usepackage{tocloft}

\setcounter{tocdepth}{2}

\renewcommand{\cftsecleader}{\cftdotfill{\cftdotsep}}
\renewcommand{\cftdotsep}{1}
\cftsetrmarg{0pt}
\renewcommand{\cfttoctitlefont}{}


\renewcommand{\cftsecpagefont}{}
\renewcommand{\cftsubsecpagefont}{}
\renewcommand{\cftsubsubsecpagefont}{}
\renewcommand{\cftparafont}{}

%\newcommand{\secnumwidth}{1cm}
%\newcommand{\subsecnumwidth}{1.5cm}
%\newcommand{\subsubsecnumwidth}{2cm}
%\newcommand{\paranumwidth}{2cm}


\setlength{\cftsecnumwidth}{2em}
\setlength{\cftsubsecnumwidth}{3em}
\setlength{\cftsubsubsecnumwidth}{4em}
\setlength{\cftparanumwidth}{5em}

\newcommand{\secindent}{0em}
\newcommand{\subsecindent}{1em}
\newcommand{\subsubsecindent}{2em}
\newcommand{\paraindent}{3em}

\setlength{\cftsecindent}{-\cftsecnumwidth}
\setlength{\cftsubsecindent}{-\cftsubsecnumwidth}
\setlength{\cftsubsubsecindent}{-\cftsubsubsecnumwidth}
\setlength{\cftparaindent}{-\cftparanumwidth}

\renewcommand{\cftsecfont}{\hspace{\cftsecnumwidth}\hspace{\secindent}}
\renewcommand{\cftsubsecfont}{\hspace{\cftsubsecnumwidth}\hspace{\subsecindent}}
\renewcommand{\cftsubsubsecfont}{\hspace{\cftsubsubsecnumwidth}\hspace{\subsubsecindent}}
\renewcommand{\cftparafont}{\hspace{\cftparanumwidth}\hspace{\paraindent}}


\setlength{\cftbeforesecskip}{0.5em}
\setlength{\cftbeforesubsecskip}{0.5em}
\setlength{\cftbeforesubsubsecskip}{0.5em}
\setlength{\cftbeforeparaskip}{0.5em}




% Файл стилизации названий разделов
% Изменение стилей названий разделов
\usepackage{titlesec}

\setcounter{secnumdepth}{4}

\titleformat{\section}[block]{\hspace{\parindent}}{\thesection}{1em}{}
\titleformat{\subsection}[block]{\hspace{\parindent}}{\thesubsection}{1em}{}
\titleformat{\subsubsection}[block]{\hspace{\parindent}}{\thesubsubsection}{1em}{}
\titleformat{\paragraph}[block]{\hspace{\parindent}}{\theparagraph}{1em}{}


\titlespacing{\section}{0pt}{2em}{2em}
\titlespacing{\subsection}{0pt}{2em}{2em}
\titlespacing{\subsubsection}{0pt}{2em}{2em}
\titlespacing{\paragraph}{0pt}{2em}{2em}

% Файл стилизации описаний рисунков и таблиц
% Изменение стилей описаний рисунков, таблиц...
\usepackage{caption}

\DeclareCaptionLabelSeparator{custom}{ -- }
\DeclareCaptionLabelFormat{flformat}{Рисунок #2}
\DeclareCaptionLabelFormat{tlformat}{Таблица #2}


\captionsetup[figure]{
	font=Large,
	labelsep=custom,
	labelformat=flformat,
	justification=centering,
	margin=1cm,
	aboveskip=0.5cm,
	belowskip=0.5cm,
}


\captionsetup[table]{
	format=plain,
	font=Large,
	labelsep=custom,
	labelformat=tlformat,
	singlelinecheck=false,
	margin=0pt,
	%margin={\docparindent, 0pt},
	aboveskip=1em,
	belowskip=0.5cm,
}

% Файл стилизации списков
% Изменение стилей списков

\usepackage{enumitem}



\newcommand{\labelv}{--}
% Отступ от label у itemize
\newcommand{\ilabelsep}{0.5em}
% Отступ от label y enumerate, level 1
\newcommand{\eilabelsep}{0.5em}
% Отступ от label у enumarate, level 2
\newcommand{\eiilabelsep}{0em}


\setlist[itemize]{
	itemsep=0pt,
	parsep=0pt,
	topsep=0pt,
	labelsep=\ilabelsep,
	label=\labelv,
	itemindent=\parindent+\ilabelsep+\labelwidth,
	leftmargin=0pt,
	align=left,
}


\setlist[enumerate,1]{
	label=\arabic*),
	ref=\arabic*,
	align=left,
	itemsep=0pt,
	parsep=0pt,
	topsep=0pt,
	labelwidth=\widthof{99)},
	labelsep=\eilabelsep,
	leftmargin=\parindent+\labelwidth+\labelsep,
}

\setlist[enumerate,2]{
	label=\theenumi.\arabic*),
	align=left,
	itemsep=0pt,
	parsep=0pt,
	topsep=0pt,
	labelsep=\eiilabelsep,
	labelwidth=\widthof{99.99)},
	leftmargin=\labelwidth+\labelsep,
}



% Файл стилизации кода

\usepackage{listings}

\lstset{
	basicstyle=\ttfamily\large,
}


%\renewcommand{\cftchapleader}{\cftdotfill{\cftdotsep}}

% Настройка отступов от краёв страницы до текста
\geometry{top = 15mm, left = 25mm, right = 10mm, bottom = 15mm}

% Убираем линию у верхнего колонтитула
\renewcommand{\headrulewidth}{0pt}

% Определяем стиль главной рамки
\fancypagestyle{mainframe} {
    \fancyhf{}
    \fancyhead[L]{
        \mainframe
        {\tpga}
        {\departmentandgroupinframe}{
        \topic
        }
        {\authorinframe}
        {\inspectorinframe}
        {\norminspectorinframe}
        {\approverinframe}
        {\thepage}
        {\total{page}}
    }

}
% Определяем стиль рамки для страниц
\fancypagestyle{pageframe} {
    \fancyhf{}
    \fancyhead[L]{
        \pageframe
        {\tpga}
        {\thepage}
    }
}


%\setlength{\textfloatsep}{1cm plus 0.5cm minus 0.5cm}

% Межстрочный интервал, равный 1.5
\onehalfspacing

% Устанавливаем отступ абзаца (Значение берется из файла констант)
\setlength{\parindent}{\docparindent}

% Выравнивание по ширине 
% (Если не влезает текст, то увеличивает отступы между словами)
\sloppy

% Регистрация счётчиков для подсчёта разных объектов документа(страниц, формул, таблиц...)
\regtotcounter{page}
\regtotcounter{equation}
\regtotcounter{figure}

% Подключение файла пользовательских стилей, команд и т.д.
\IfFileExists{custom.tex}{
    \input{custom.tex}
}{
    % Здесь указываются пользовательские стили, команды...
% Также возможно преопределение стилей, команд


% Команда изменения стиля темы в рамке.
% В данном примере размер шрифта - large.
% Если данная команда не будет указана, то по умолчанию 
% для темы в рамке будет указан размер шрифта - \large.
% Если вам не нужно менять стиль темы, то закомментируйте или удалите данную команду.
\newcommand{\topicstyleinframe}{\large}

}


% Начало документа
\begin{document}
% Меняем название заголовка таблицы содержимого
\renewcommand{\contentsname}{\hfill Содержание \hfill}
% Устанавливаем размер шрифта 14pt

\Large
% Устанавливаем стиль пустой, чтобы убрать дефолтную нумерацию страниц
\pagestyle{empty}
% Подключаем титульник
% Титульник не изменяет счётчик страниц, он как бы не участвует в нумерации страниц,
% поэтому нумерация страниц начнётся со страницы "Реферат". 
% В итоге как бы реферат не будет участвовать в нумерации, что нам и нужно
\IfFileExists{./pages/title.tex} {
	\begin{titlepage}
	\newpage
	\singlespacing
	\topskip = 0.8cm
	\large
	%\setlength{\ULdepth}{1.8pt}

	\begin{center}
		\large
		%\fontsize{11pt}{13pt}
		\bfseries
		МИНИСТЕРСТВО НАУКИ И ВЫСШЕГО ОБРАЗОВАНИЯ РФ \\
		ФЕДЕРАЛЬНОЕ ГОСУДАРСТВЕННОЕ БЮДЖЕТНОЕ \\
		ОБРАЗОВАТЕЛЬНОЕ УЧРЕЖДЕНИЕ ВЫСШЕГО ОБРАЗОВАНИЯ \\
		«ВЯТСКИЙ ГОСУДАРСТВЕННЫЙ УНИВЕРСИТЕТ» \\
		ИНСТИТУТ МАТЕМАТИКИ И ИНФОРМАЦИОННЫХ СИСТЕМ \\
		ФАКУЛЬТЕТ АВТОМАТИКИ И ВЫЧИСЛИТЕЛЬНОЙ ТЕХНИКИ \\
		КАФЕДРА ЭЛЕКТРОННЫХ ВЫЧИСЛИТЕЛЬНЫХ МАШИН
	\end{center}

	\vspace{0.8cm}
	\begin{center}
		\textbf{Направление}
		\uline{\specialization}

		\small
		\textit{(код и наименование направления)}

		\large
		Профиль – \uline{\profile}
	\end{center}

	\vspace{0.8cm}
	\begin{flushright}
		Допускаю к защите \\
		Заведующий кафедрой ЭВМ \\
		\vspace{1mm}
		\uline{\hspace{3cm}} / \uline{\headofdepartment} / \\
		\vspace{1mm}
		\small
		\itshape
		(подпись) \hspace{1.8cm} (Ф.И.О.) \hspace{1.4cm}

	\end{flushright}

	\vspace{1.5cm}
	\begin{center}
		\huge
		\bfseries
		\topic
	\end{center}

	\begin{center}
		Пояснительная записка выпускной квалификационной работы \\
		\tpga
	\end{center}

	\newcommand{\ulinesize}{2.5cm}

	\large
	\vspace{1cm}
	\noindent
	Разработал: \studentgroup \hfill \uline{\hspace{\ulinesize}}
	/ \uline{\authorwithinitials} / \hspace{8mm} \uline{\hspace{\ulinesize}}

	\vspace{1.5cm}
	\noindent
	Руководитель: \supervisorrank
	\hfill \uline{\hspace{\ulinesize}}
	/ \uline{\supervisor} / \uline{\hspace{\ulinesize}}

	\vspace{1.5cm}
	\noindent
	Нормоконтролер: \norminspectorrank
	\hfill \uline{\hspace{\ulinesize}}
	/ \uline{\norminspector} / \hspace{4mm} \uline{\hspace{\ulinesize}}

	{
		\small
		\itshape
		\hfill
		(подпись) \hspace{1.6cm} (Ф.И.О) \hspace{2.2cm} (дата) \hspace{0.8cm}
	}

	\begin{center}
		\vfill
		Киров \the\year
		\vspace{1cm}
	\end{center}


\end{titlepage}

}{
	\begin{titlepage}

	\newcommand{\verticalspace}{1.5cm}


	\ifthenelse{\boolean{consultantexists}}{
		\renewcommand{\verticalspace}{1.2cm}}{}

	\newpage
	\singlespacing
	\vspace{0.8cm}
	\large
	%\setlength{\ULdepth}{1.8pt}

	\begin{center}
		\large
		%\fontsize{11pt}{13pt}
		\bfseries
		МИНИСТЕРСТВО НАУКИ И ВЫСШЕГО ОБРАЗОВАНИЯ РФ \\
		ФЕДЕРАЛЬНОЕ ГОСУДАРСТВЕННОЕ БЮДЖЕТНОЕ \\
		ОБРАЗОВАТЕЛЬНОЕ УЧРЕЖДЕНИЕ ВЫСШЕГО ОБРАЗОВАНИЯ \\
		«ВЯТСКИЙ ГОСУДАРСТВЕННЫЙ УНИВЕРСИТЕТ» \\
		ИНСТИТУТ МАТЕМАТИКИ И ИНФОРМАЦИОННЫХ СИСТЕМ \\
		ФАКУЛЬТЕТ АВТОМАТИКИ И ВЫЧИСЛИТЕЛЬНОЙ ТЕХНИКИ \\
		КАФЕДРА ЭЛЕКТРОННЫХ ВЫЧИСЛИТЕЛЬНЫХ МАШИН
	\end{center}

	\vspace{0.8cm}
	\begin{center}
		\textbf{Направление}
		\uline{\specialization}

		\small
		\textit{(код и наименование направления)}

		\large
		Профиль – \uline{\profile}
	\end{center}

	\vspace{0.8cm}
	\begin{flushright}
		{
			\onehalfspacing
			Допускаю к защите \\
			Заведующий кафедрой ЭВМ \\
		}
		\vspace{1mm}
		\uline{\hspace{3cm}} / \uline{\headofdepartment} / \\
		\vspace{1mm}
		\small
		\itshape
		(подпись) \hspace{1.8cm} (Ф.И.О.) \hspace{1.4cm}

	\end{flushright}

	\vspace{1.5cm}
	\begin{center}
		\huge
		\bfseries
		\topic
	\end{center}
	\vspace{0pt}
	\begin{center}
		Пояснительная записка выпускной квалификационной работы \\
		\tpga
	\end{center}

	\newcommand{\ulinesize}{2.5cm}
	\newcommand{\namesize}{3.7cm}

	\large
	\vspace{1cm}
	\noindent
	Разработал: \studentgroup
	\hfill
	\uline{\hspace{\ulinesize}}
	/\uline{\makebox[\namesize][l]{ \authorwithinitials}} /
	\uline{\hspace{\ulinesize}}

	\vspace{\verticalspace}
	\noindent
	Руководитель: \supervisorrank
	\hfill
	\uline{\hspace{\ulinesize}}
	/\uline{\makebox[\namesize][l]{ \supervisor}} /
	\uline{\hspace{\ulinesize}}

	\ifthenelse{\boolean{consultantexists}}{
		\vspace{\verticalspace}
		\noindent
		Консультант: \consultantrank
		\hfill
		\uline{\hspace{\ulinesize}}
		/\uline{\makebox[\namesize][l]{ \consultant}} /
		\uline{\hspace{\ulinesize}}}{}

	\vspace{\verticalspace}
	\noindent
	Нормоконтролер: \norminspectorrank
	\hfill
	\uline{\hspace{\ulinesize}}
	/\uline{\makebox[\namesize][l]{ \norminspector}} /
	\uline{\hspace{\ulinesize}}

	{
		\small
		\itshape
		\hfill
		\makebox[\ulinesize][c]{(подпись)}~
		\makebox[\namesize][c]{(Ф.И.О)}~
		\makebox[\ulinesize][c]{(дата)}
	}

	\begin{center}
		\vfill
		Киров \the\year
		\vspace{1cm}
	\end{center}


\end{titlepage}

}

% Меняем размер нижнего отступа до текста, чтобы текст не заезжал на главную рамку. 
% (Отступ текста до рамки 10mm)
% Меняем как бы на странице "Реферат", чтобы изменения были задействованы на странице 
% "Содержание". 
\addtolength{\textheight}{-40mm}

% Подключаем страницу "Реферат"
\IfFileExists{./pages/abstract.tex} {
	
{

\topskip = 0.8cm
\begin{center}
	Реферат
\end{center}

\vspace{1em}

\authori\
\topic
:\
\mbox{\tpga}\
ВКР / ВятГУ, каф. ЭВМ; рук.
Долженкова М.Л. – Киров, \the\year. –
Гр.ч. 8 л. ф.А1;
ПЗ
\total{page} с.,
\total{figure} рис.,
1 табл.,
\total{equation} форм.,
1 источников,
1 прил.

\vspace{1.5em}


МАШИНА ВРЕМЕНИ,
МАШИНА, ВРЕМЕНИ,
БУДУЩЕЕ, ПРОШЛОЕ,
УСТРОЙСТВО,
C++,
C.


\vspace{1.5em}

Объект - Машина времени, устройство или технология, позволяющая перемещаться во времени, как в прошлое, так и в будущее.

Цель - Создание машины времени имеет множество возможных целей, включая исследование и изучение исторических событий, изменение хода истории, исправление ошибок или предотвращение негативных событий, получение информации из будущего для решения проблем настоящего, и т.д.

Результат - Результат создания машины времени может быть разным в зависимости от целей, методов и последствий использования. Возможны различные сценарии, такие как изменение прошлого с созданием параллельных временных линий, возможные временные парадоксы, потенциальные изменения будущего и т.д. Создание машины времени представляет сложную и фантастическую тему, вызывающую много вопросов и интерпретаций.



}

}{
	
{

\vspace{0.8cm}
\begin{center}
	Реферат
\end{center}

\vspace{1em}

\authorwithinitials\
\topic
:\
\mbox{\tpga}\
ВКР / ВятГУ, каф. ЭВМ; рук.
\supervisor – Киров, \the\year. –
Гр.ч. \numberofposters л. ф.А1;
ПЗ
\total{page} с.,
\total{figure} рис.,
\total{table} табл.,
\total{equation} форм.,
\ref*{ref:total} источников,
\total{appendix} прил.

\vspace{1.5em}

\IfFileExists{pages/abstract/tags.tex}{
	
МАШИНА ВРЕМЕНИ,
МАШИНА, ВРЕМЕНИ,
БУДУЩЕЕ, ПРОШЛОЕ,
УСТРОЙСТВО,
C++,
C.

}{
	
МАШИНА ВРЕМЕНИ,
МАШИНА, ВРЕМЕНИ,
БУДУЩЕЕ, ПРОШЛОЕ,
УСТРОЙСТВО,
C++,
C.

}

\vspace{1.5em}

\IfFileExists{pages/abstract/content.tex}{
	Объект - Машина времени, устройство или технология, позволяющая перемещаться во времени, как в прошлое, так и в будущее.

Цель - Создание машины времени имеет множество возможных целей, включая исследование и изучение исторических событий, изменение хода истории, исправление ошибок или предотвращение негативных событий, получение информации из будущего для решения проблем настоящего, и т.д.

Результат - Результат создания машины времени может быть разным в зависимости от целей, методов и последствий использования. Возможны различные сценарии, такие как изменение прошлого с созданием параллельных временных линий, возможные временные парадоксы, потенциальные изменения будущего и т.д. Создание машины времени представляет сложную и фантастическую тему, вызывающую много вопросов и интерпретаций.

} {
	Объект - Машина времени, устройство или технология, позволяющая перемещаться во времени, как в прошлое, так и в будущее.

Цель - Создание машины времени имеет множество возможных целей, включая исследование и изучение исторических событий, изменение хода истории, исправление ошибок или предотвращение негативных событий, получение информации из будущего для решения проблем настоящего, и т.д.

Результат - Результат создания машины времени может быть разным в зависимости от целей, методов и последствий использования. Возможны различные сценарии, такие как изменение прошлого с созданием параллельных временных линий, возможные временные парадоксы, потенциальные изменения будущего и т.д. Создание машины времени представляет сложную и фантастическую тему, вызывающую много вопросов и интерпретаций.

}

}

}

% Устанавливаем главную рамку для одной страницы
\tocloftpagestyle{mainframe}

% Устанавливаем рамку страниц, которая будет отрисовываться после главной
\pagestyle{pageframe}

% Меняем размер нижнего отступа до текста, чтобы текст до рамки страниц был 10mm.
% Рамка страниц имеет меньшую высоту содержимого нижней таблицы.
\addtolength{\textheight}{+25mm}
% Устанавливаем "Содержание", включающее в себя описание разделов документа
\tableofcontents\newpage

% Подключаем содержимое документа
\IfFileExists{./pages/content.tex} {
	Объект - Машина времени, устройство или технология, позволяющая перемещаться во времени, как в прошлое, так и в будущее.

Цель - Создание машины времени имеет множество возможных целей, включая исследование и изучение исторических событий, изменение хода истории, исправление ошибок или предотвращение негативных событий, получение информации из будущего для решения проблем настоящего, и т.д.

Результат - Результат создания машины времени может быть разным в зависимости от целей, методов и последствий использования. Возможны различные сценарии, такие как изменение прошлого с созданием параллельных временных линий, возможные временные парадоксы, потенциальные изменения будущего и т.д. Создание машины времени представляет сложную и фантастическую тему, вызывающую много вопросов и интерпретаций.

}{
	% Привер файла содержимого документа
% Измените подключаемые файлы в зависимости от вашей структуры документа


\csection{Введение}

Всеобщее стремление человечества к исследованию пространства и времени привело к поиску
способов путешествия во времени. Разработка машины времени – это
одна из самых захватывающих и сложных задач на пути человечества
к познанию вселенной. Понимание сущности времени и его влияния
на нашу жизнь заставляют нас задуматься о возможности изменения прошлого
и будущего. Несмотря на фантастический характер такой концепции,
многие ученые вкладывают огромные усилия в поиск способов создания устройства,
способного перемещать нас во времени.

В данном исследовании мы рассмотрим основные
принципы и технологии, необходимые для разработки машины времени, а также потенциальные последствия и этические вопросы, связанные с реализацией подобной технологии. Впереди нас ждут захватывающие открытия и новые горизонты познания времени и пространства.

\pagebreak


\section{Анализ предметной области}

Машина времени является гипотетическим устройством, способным перемещаться во времени, обеспечивая возможность путешествия в прошлое или будущее. В связи с этим, анализ предметной области машины времени включает следующие аспекты:


\begin{enumerate}
	\item физические принципы: необходимо изучить теоретические основы, согласно которым могла бы функционировать машина времени. Это может включать обсуждение специальной теории относительности, черных дыр, петель времени и других концепций из физики;

	\item технические особенности: рассмотрим возможные способы построения и дизайна машины времени. Какие технологии или материалы могут быть использованы для ее создания? Какие опасности или препятствия могут возникнуть при ее конструировании?

	\item парадоксы времени: проанализируем различные парадоксы, которые могут возникнуть при использовании машины времени, такие как "парадокс дедушки" или "парадокс самовыполнения". Какие последствия они могут иметь для временных путешественников?

	\item этические и социальные аспекты: обсудим влияние машины времени на общество и индивидуума. Какие этические вопросы возникают при использовании данной технологии? Какие последствия она может иметь для личной и мировой истории?

	\item возможные приложения: рассмотрим потенциальные области применения машины времени, такие как исследования истории, изменение прошлого или будущего, предсказание событий и т. д.
\end{enumerate}


\section*{Выводы}


Анализ предметной области машины времени требует не только фантазии и творческого мышления, но и глубоких знаний в области физики, философии, этики и технологии.



\newpage

\section{Расширенное техническое задание}

В данном разделе представлено техническое задание на разработку машины времени.


\subsection{Введение}
Цель проекта - разработать машину времени, способную перемещаться в прошлое и будущее для исследования и изменения событий.

\subsection{Требования к функциональности}
\begin{itemize}
	\item машина времени должна быть способна точно управлять перемещением во времени и пространстве;
	\item должна быть возможность указания точного временного и пространственного пункта назначения;
	\item необходим механизм возврата в исходную точку времени и пространства;
	\item должна обеспечиваться безопасность для пассажиров и времени путешествия;
	\item возможность регистрации событий и данных во время путешествия во времени.
\end{itemize}

\subsection{Требования к проектированию}

\begin{itemize}
	\item разработка основывается на принципах теории относительности и квантовой механики;
	\item использование передовых материалов и технологий для обеспечения надежности и функциональности машины времени;
	\item должен быть предусмотрен механизм защиты от парадоксов временных петель.
\end{itemize}

\section*{Выводы}

Составлено техническое задание.

\newpage

\section{Структура машины времени}

Структура машины времени представлена на рисунке~\ref{f:time-machine}.


\begin{figure}[ht]
	\centering
	\vspace{\toppaddingoffigure}
	\includegraphics[width=0.7\textwidth]{time-machine}
	\caption{Структура машины времени}
	\label{f:time-machine}
\end{figure}


Разработка структуры машины времени является сложной задачей, поскольку такое устройство находится за пределами существующих научных и инженерных возможностей. Однако, если мы предположим, что машина времени возможна, то ее структура, вероятно, будет иметь следующие основные компоненты:

\begin{enumerate}
	\item часовой механизм: стандартный механизм, который управляет передвижением во времени, аналогично тому, как часовой механизм контролирует передвижение стрелок на циферблате часов;

	\item энергетический источник: мощный источник энергии, способный обеспечить работу машины времени и перемещение объектов во времени;

	\item контрольная система: комплекс алгоритмов и программного обеспечения, которые контролируют точное время и координируют перемещение во времени;

	\item защитные механизмы: системы, предотвращающие нежелательное перемещение во времени или обеспечивающие безопасность при использовании машины времени;

	\item интерфейс пользователя: устройства ввода и вывода, позволяющие пользователю программировать желаемые временные точки или координировать перемещение во времени;

	\item материалы и конструкция: специальные материалы и компоненты, обеспечивающие устойчивость и работоспособность машины времени.
\end{enumerate}

Хотя это очень упрощенное описание возможной структуры машины времени, это может помочь представить основные компоненты, которые потребуются для того, чтобы создать такое устройство. Однако необходимо помнить, что вышеописанная концепция является вымышленной и не имеет научного обоснования.


Пример ссылки на источник \refref{ref:num-methods}.

Пример еще ссылки на источник \refref{ref:time-series-analysis}.

\subsection{Расчёты}

Расчёты представлены ниже

\begin{gather}
	a = \tan(\frac{\alpha}{2})*a*\pi, \\
	\bigtriangleup b = \cos(\beta)*a, \\
	c = \sin{\beta}.
\end{gather}


\examplecommand


\docappendix{Листинг кода}\label{ax:code}

\begin{lstlisting}
var a = 1;
var b = 10;
var c = 123;
var d = a + b + c;

some_function(a, b, c);
\end{lstlisting}


\docappendix{Авторская справка}\label{ax:authornote}
\begin{singlespace}

	\newcommand{\authorfullname}{Волков Михаил Владимировч}
	\renewcommand{\authorwithinitials}{Волков М. В.}
	\renewcommand{\headofdepartment}{М. Л. Долженкова}

	Я, \authorfullname , автор выпускной квалификационной
	работы «\topic» сообщаю, что мне известно о персональной ответственности автора
	за разглашение сведений, подлежащих защите законами РФ о защите объектов
	интеллектуальной собственности.

	Одновременно сообщаю, что:

	1. При подготовке к защите выпускной квалификационной работы не
	использованы источники (документы, отчёты, диссертации, литература и т.п.),
	имеющие гриф секретности или «Для служебного пользования» ФГБОУ ВО
	«Вятский государственный университет» или другой организации.

	2. Данная работа не связана с незавершёнными исследованиями или уже
	с завершёнными, но ещё официально не разрешёнными к опубликованию
	ФГБОУ ВО «Вятский государственный университет» или другими
	организациями.

	3. Данная работа не содержит коммерческую информацию, способную
	нанести ущерб интеллектуальной собственности ФГБОУ ВО «Вятский
	государственный университет» или другой организации.

	4. Данная работа не является результатом НИР или ОКР, выполняемой по
	договору с организацией.

	5. В предлагаемом к опубликованию тексте нет данных по
	незащищённым объектам интеллектуальной собственности других авторов.

	6. Использование моей дипломной работы в научных исследованиях
	оформляется в соответствии с законодательством РФ о защите интеллектуальной
	собственности отдельным договором.

	\newcommand{\daysize}{2em}

	\newcommand{\confirm}{
		\noindent
		Сведения по авторской справке подтверждаю:~«\uline{\hspace{\daysize}}»
	}

	\newcommand{\verticalspacesize}{2em}

	\newcommand{\rightindent}{2em}

	\newcommand{\customtheyear}{\the\year~г.}

	\newcommand{\rightindentwithyear}{\widthof{\customtheyear}+\rightindent}

	\newcommand{\signaturesize}{6.6em}

	\vspace{\verticalspacesize}


	\noindent
	Автор: \authorwithinitials~«\uline{\hspace{\daysize}}»
	\uline{\hspace{6em}}
	\customtheyear
	\hfill
	\uline{\hspace{\signaturesize}}\hspace{\rightindentwithyear}

	\vspace{-0.3em}
	\hfill
	\makebox[\signaturesize][c]{\small подпись}
	\hspace{\rightindentwithyear}

	\vspace{\verticalspacesize}

	\confirm
	\uline{\hfill}\customtheyear\hspace{\rightindent}

	\vspace{\verticalspacesize}

	\noindent
	Заведующая кафедрой ЭВМ: \headofdepartment
	\hfill
	\uline{\hspace{\signaturesize}}\hspace{\rightindentwithyear}

	\vspace{-0.3em}
	\hfill
	\makebox[\signaturesize][c]{\small подпись}
	\hspace{\rightindentwithyear}

\end{singlespace}



\docappendix[справочное]{Какое-то справочное приложение}\label{ax:ref}


Содержимое приложения


}


\end{document}
