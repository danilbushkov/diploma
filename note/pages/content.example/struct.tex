\newpage

\section{Структура машины времени}

Структура машины времени представлена на рисунке~\ref{f:time-machine}.


\begin{figure}[ht]
	\centering
	\vspace{\toppaddingoffigure}
	\includegraphics[width=0.7\textwidth]{time-machine}
	\caption{Структура машины времени}
	\label{f:time-machine}
\end{figure}


Разработка структуры машины времени является сложной задачей, поскольку такое устройство находится за пределами существующих научных и инженерных возможностей. Однако, если мы предположим, что машина времени возможна, то ее структура, вероятно, будет иметь следующие основные компоненты:

\begin{enumerate}
	\item часовой механизм: стандартный механизм, который управляет передвижением во времени, аналогично тому, как часовой механизм контролирует передвижение стрелок на циферблате часов;

	\item энергетический источник: мощный источник энергии, способный обеспечить работу машины времени и перемещение объектов во времени;

	\item контрольная система: комплекс алгоритмов и программного обеспечения, которые контролируют точное время и координируют перемещение во времени;

	\item защитные механизмы: системы, предотвращающие нежелательное перемещение во времени или обеспечивающие безопасность при использовании машины времени;

	\item интерфейс пользователя: устройства ввода и вывода, позволяющие пользователю программировать желаемые временные точки или координировать перемещение во времени;

	\item материалы и конструкция: специальные материалы и компоненты, обеспечивающие устойчивость и работоспособность машины времени.
\end{enumerate}

Хотя это очень упрощенное описание возможной структуры машины времени, это может помочь представить основные компоненты, которые потребуются для того, чтобы создать такое устройство. Однако необходимо помнить, что вышеописанная концепция является вымышленной и не имеет научного обоснования.


Пример ссылки на источник \refref{ref:num-methods}.

Пример еще ссылки на источник \refref{ref:time-series-analysis}.

\subsection{Расчёты}

Расчёты представлены ниже

\begin{gather}
	a = \tan(\frac{\alpha}{2})*a*\pi, \\
	\bigtriangleup b = \cos(\beta)*a, \\
	c = \sin{\beta}.
\end{gather}


\examplecommand
